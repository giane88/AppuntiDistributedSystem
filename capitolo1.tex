\label{capitolo1}
\section{Introduzione}
A partire dalla metà degli anni '80, grazie a due innovazioni tecnologiche si fecero diversi passi avanti nell'uso dei calcolatori. La prima di queste innovazioni fu lo sviluppo di microprocessori potenti; la seconda grande innovazione fu l'invenzione delle reti di computer con l'introduzione delle \textbf{LAN} (\emph{Local Area Network}) che consentirono a centinaia di macchine di essere connesse le une alle altre e permisero lo scambio di piccole quantità di informazioni in pochi microsecondi.\\
Il risultato di questa innovazione tecnologica è che oggi mettere insieme una grande quantità di computer tramite una rete ad alta velocità è diventato molto semplice. Questo tipo di sistemi sono solitamente chiamate \emph{reti di computer} o \textbf{sistemi distribuiti}.
\subsection{Definizione di sistema distribuito}
Esistono diverse definizioni di \emph{Sistema distribuito} ma tutte quante sono abbastanza insoddisfacenti. Daremo ora una prima definizione che è sufficente per i nostri scopi:
\begin{center}
\emph{Un sistema distribuito è una collezione di computer indipendenti che appare ai propri utenti come un singolo sistema coerente}
\end{center}
Da questa definizione possiamo ricavare diversi caratteristiche di un sistema distribuito, la prima è che i sistemi distribuiti sono costituiti da componenti autonomi; la seconda è che gli utenti, siano essi persone o altri programmi, vedono il sistema come un'unica entità. Il che significa che i diversi componenti devono in qualche modo collaborare.\\
Quello che non viene specificato in questa definizione è il tipo di computer usati per i componenti ne come questi sono interconnessi.\\
Le caratteristiche più importanti dei sistemi distribuiti sono il fatto che le differenze tra i vari computer e le loro modalità di comunicazione risultano per lo più nascoste agli utenti finali. Inoltre gli utenti possono interagire con un sistema distribuito in modo \emph{consistente} e \emph{uniforme} ovvero indipendentemente da dove e quando l'interazione ha luogo.